% sections/03-methodology.tex
% 研究方法

\section{系统方案}

\begin{frame}[t]{系统方案:LangGraph + RAG 的工程化落地}
    \framepills{\pillaccent{Architecture}\hspace{0.15cm}\pill{Next.js}\hspace{0.15cm}\pill{Java Gateway}\hspace{0.15cm}\pill{LangGraph}\hspace{0.15cm}\pill{RAG}}
    \vspace{-0.30cm}

    \begin{columns}[T, onlytextwidth]
        \begin{column}{0.54\textwidth}
            \begin{aicard}[Components]
                \begin{itemize}\setlength{\itemsep}{0.16cm}
                    \item \textbf{前端(Next.js)}:对话 UI、服务类型入口、SSE 流式渲染
                    \item \textbf{后端(Java)}:JWT 鉴权、会话管理、审计留痕、限流/风控
                    \item \textbf{智能体(FastAPI + LangGraph)}:多步推理 + 工具调用,必要时检索引用\cite{langgraph2024}
                    \item \textbf{数据层}:向量检索(Cloudflare Vectorize)+ 文件存储(R2)+ 结构化日志(DB)
                \end{itemize}
                {\footnotesize\color{muted}目标:把“咨询”变成可追溯的工程流程。}
            \end{aicard}
        \end{column}
        \begin{column}{0.46\textwidth}
            \begin{aicard}[Flow]
                \centering
                \begin{tikzpicture}[node distance=0.58cm, every node/.style={font=\fontsize{8.2}{10.2}\selectfont, align=center}]
                    \node[rounded corners=3.2mm, draw=line, fill=bg, line width=0.55pt, minimum width=5.05cm, minimum height=0.80cm] (fe) {\textbf{Frontend}\\ Next.js / Chat UI};
                    \node[rounded corners=3.2mm, draw=line, fill=bg, line width=0.55pt, minimum width=5.05cm, minimum height=0.80cm, below=0.28cm of fe] (be) {\textbf{Backend}\\ Auth / Session / Audit};
                    \node[rounded corners=3.2mm, draw=line, fill=bg, line width=0.55pt, minimum width=5.05cm, minimum height=0.80cm, below=0.28cm of be] (ai) {\textbf{Agent}\\ LangGraph + Tools};
                    \node[rounded corners=3.2mm, draw=line, fill=bg, line width=0.55pt, minimum width=5.05cm, minimum height=0.80cm, below=0.28cm of ai] (data) {\textbf{Data}\\ Vectorize / R2 / DB};

                    \draw[->, line width=0.85pt, color=accent] (fe) -- (be);
                    \draw[->, line width=0.85pt, color=accent] (be) -- (ai);
                    \draw[->, line width=0.85pt, color=accent] (ai) -- (data);
                \end{tikzpicture}
            \end{aicard}
        \end{column}
    \end{columns}
\end{frame}

\begin{frame}[t, fragile]{关键流程:工具调用 + 引用 + 追溯(traceId)}
    \framepills{\pillaccent{Trace}\hspace{0.15cm}\pill{Tools}\hspace{0.15cm}\pill{Citations}\hspace{0.15cm}\pill{Replayable}}

    \begin{columns}[T, onlytextwidth]
        \begin{column}{0.54\textwidth}
            \begin{aicard}[Request pipeline]
                \begin{itemize}\setlength{\itemsep}{0.24cm}
                    \item \textbf{Input}:问题/合同片段(可选脱敏)+ 用户身份
                    \item \textbf{Guard}:鉴权、限流、风险提示模板
                    \item \textbf{Agent}:决策(检索/生成/追问)
                    \item \textbf{Tools}:RAG / OCR / 文书生成
                    \item \textbf{Output}:建议 + 免责声明 +(可选)引用 + traceId
                \end{itemize}
            \end{aicard}
        \end{column}
        \begin{column}{0.46\textwidth}
            \begin{aicard}[LangGraph (pseudo)]
                \lstset{language=Python, basicstyle=\ttfamily\scriptsize, numbers=none, frame=none, backgroundcolor=\color{bg}, linewidth=\linewidth, xleftmargin=0pt, xrightmargin=0pt}
                \begin{lstlisting}
graph = StateGraph(State)
graph.add_node("decide", decide)
graph.add_node("rag", legal_rag_search)
graph.add_edge("decide", "rag")
graph.add_edge("rag", "decide")
graph.compile()
                \end{lstlisting}
                {\footnotesize\color{muted}要点:把多步推理固定为图结构,便于测试与回放。}
            \end{aicard}
        \end{column}
    \end{columns}
\end{frame}
