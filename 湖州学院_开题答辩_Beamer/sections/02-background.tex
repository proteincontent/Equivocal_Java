% sections/02-background.tex
% 研究背景

\section{背景与问题}

\begin{frame}[c]{背景与意义:为什么需要法律智能体}
    \framepills{\pillaccent{Problem}\hspace{0.15cm}\pill{Cost}\hspace{0.15cm}\pill{Risk}\hspace{0.15cm}\pill{Fragmented info}}

    \begin{columns}[T, onlytextwidth]
        \begin{column}{0.58\textwidth}
            \begin{aicard}[User pain]
                \begin{itemize}\setlength{\itemsep}{0.32cm}
                    \item \textbf{成本高、门槛高}:依赖专业人力;用户常常“问不到点子上”
                    \item \textbf{风险点隐蔽}:条款风险埋在细节里,误解可能带来高代价
                    \item \textbf{信息碎片化}:法条/案例/流程分散,检索与比对成本高
                \end{itemize}
            \end{aicard}
        \end{column}
        \begin{column}{0.42\textwidth}
            \begin{aicard}[Why now]
                \textbf{大模型}带来自然语言交互能力,\textbf{RAG}把回答与依据绑定,让关键结论\textbf{可追溯、可复核}\cite{lewis2020rag}。\par
                \vspace{0.35em}
                {\footnotesize\color{muted}目标不是“更会聊天”,而是“更可验收”。}
            \end{aicard}
        \end{column}
    \end{columns}
\end{frame}

\begin{frame}[c]{问题定义:把“咨询”工程化到可验收}
    \framepills{\pillaccent{Definition}\hspace{0.15cm}\pill{MVP}\hspace{0.15cm}\pill{Traceable}\hspace{0.15cm}\pill{Safe boundary}}

    \begin{columns}[T, onlytextwidth]
        \begin{column}{0.52\textwidth}
            \begin{aicard}[One‑sentence mission]
                让用户在 3 分钟内得到\textbf{可执行的下一步}(需要补充哪些事实 / 风险点在哪里 / 建议怎么做),并保证\textbf{可追溯、可回放、可验收}。
            \end{aicard}
        \end{column}
        \begin{column}{0.48\textwidth}
            \begin{aicard}[Boundaries \& principles]
                \begin{itemize}\setlength{\itemsep}{0.26cm}
                    \item \textbf{不替代律师}:输出建议与依据,提示适用范围与不确定性
                    \item \textbf{可追溯}:关键输出绑定 traceId / 模型信息 / 引用来源
                    \item \textbf{可降级}:上游不可用时返回结构化替代步骤(非“报错/沉默”)
                    \item \textbf{隐私最小化}:最小化存储、可选脱敏,避免泄露敏感信息
                \end{itemize}
            \end{aicard}
        \end{column}
    \end{columns}
\end{frame}
